\documentclass{article}
\usepackage[dvips]{graphicx}
\usepackage{amsmath}
\usepackage{amssymb}
\usepackage{amsthm}

\title{Metodos numericos}
\author{Julio C. Estrada}

\begin{document}
\maketitle

\section{Conceptos b\'asicos}
\subsection{Derivada}
La derivada de una funcion $f(x)$ continua es definida de la siguiente
manera:
\begin{equation}
  f(x)=\lim_{h\to 0}\frac{f(x+h)-f(x)}{h}=f'(x)=\frac{dx}{dy},
\end{equation}
suponiendo que el l\'imite existe. Otras formas de expresar la
derivada:
\begin{equation}
  \frac{dx}{dy}=\lim_{\Delta x\to 0}\frac{\Delta y}{\Delta x}\quad
  \text{\'o}
  \quad \frac{dy}{dx}_{x=x_0}=f'(x_0)=\lim_{x\to x_0}\frac{f(x)-f(x_0)}{x-x_0}
\end{equation}

\subsection{Teorema fundamental del calculo}
Sea $F(x)$ una funcion continua en el intervalo $(a,b)$ tal que
$\frac{dF(x)}{dx}=f(x)$. La funcion $f(x)$ es una funcion continua
tambien dentro del intervalo $(a,b)$. Ahora divdiamos el intervalo en
$n$ subintervalos de tama\~no $\Delta x=\frac{b-a}{n}$. Tomando
$a=x_0$ y $b=x_n$, el intervalo quedaria partido en $n$ subintervalos
cuys posiciones $i$-esima seria $x_i=x_0+i\Delta x$ para
$i=0,1,2,\dots,n$. Ahora digamos que el valor $c_i$ es un valor dentro
del intervalo $[x_{i-1},x_i]$ y formemos la siguiente suma
\begin{equation}
  S_n=\sum_{i=0}^{n}f(c_i)\Delta x
\end{equation}
El \emph{Teorema fundamental del calculo} nos dice que
\begin{equation}
\lim_{n\to \infty}S_n=\lim_{n\to \infty}\sum_{i=0}^{n}f(c_i)\Delta x =
\int_a^bf(x)dx
= \left. F(x)\right]_a^b=F(b)-F(a)
\end{equation}
Si $G(x)=\int_a^xf(t)dt$, entonces $\frac{dG(x)}{dx}=f(x)$

\subsection{Series de Taylor}
Una funcion $f(x)$ se dice que es anal\'itica en la vecindad de
$x=x_0$ si puede ser representada como una serie de potencias
convergente de la siguiente manera:
\begin{equation}\label{eq:taylor}
\begin{split}
f(x)&=f(x_0)+f'(x_0)(x-x_0) + \frac{f''(x_0)}{2!}(x-x_0)^2+\cdots +
\frac{f^{(m)}(x_0)}{m!}(x-x_0)^m + \cdots \\
 &=\sum_{n=0}^{\infty}\frac{f^{(n)}(x_0)}{n!}(x-x_0)^n
\end{split}
\end{equation}
Aqui suponemos que la funcion tiene derivadas continuas en todos los
ordenes que puede ser evaluado el punto $x_0$. A la
Ec. \ref{eq:taylor} le llamamos serie de Taylor. Otras maneras de
expresar la serie de Taylor son las siguientes:
\begin{equation}
\begin{split}
f(x_0+h) &= f(x_0) + f'(x_0)h + f''(x_0)\frac{h^2}{2!}+ \cdots +
f^{m}(x_0)\frac{h^m}{m!} + \cdots \\
  &=\sum_{n=0}^{\infty}f^{(n)}(x_0)\frac{h^n}{n!}
\end{split}
\end{equation}

Algunas expanciones en series de taylor son las siguientes:
\begin{equation}
\begin{split}
e^x &=1+x+\frac{x^2}{2!} + \frac{x^3}{3!} + \frac{x^4}{4!} +\cdots\\
\sin x &=x - \frac{x^3}{3!}+ \frac{x^5}{5!} - \frac{x^7}{7!} + \cdots\\
\cos x &=1-\frac{x^2}{2!} + \frac{x^4}{4!} + \frac{x^6}{6!} + \cdots
\end{split}\nonumber
\end{equation}
La serie de Taylor truncada es expresada de la siguiente manera:
\begin{equation}
f(x) = =f(x_0)+f'(x_0)(x-x_0) + \frac{f''(x_0)}{2!}(x-x_0)^2+\cdots +
\frac{f^{(m)}(x_0)}{m!}(x-x_0)^m + Error
\end{equation}
donde el error viene dado por
\begin{equation}
Error = \frac{f^{(m+1)}(\xi)}{(m+1)!}(x-x_0)^{m+1},\quad x_0<\xi<x
\end{equation}
Decimos que una funcion $f(x)$ tiene una raiz de multiplicidad $m$ si
\begin{equation}
f(x_0)=f'(x_0)=f''(x_0)=\cdots = f^{(m-1)}(x_0)=0
\end{equation}
Y su serie de Taylor queda de la siguiente manera:
\begin{equation}
f(x_0+h)=f^{(m)}(x_0)\frac{h}{m!}+f^{m+1}(x_0)\frac{h^{m+1}}{(m+1)!}+\cdots
\end{equation}

\subsection{El simbolo Landau $\mathcal O$}
El simbolo Landau $\mathcal O$ es usado para comparar el
comportamiento de una funcion $f(h)$ con otra $g(h)$ cuando $h\to
0$. Decimos que 
\begin{equation}
f(h)=\mathcal{O}(g(h))\quad \text{si}\quad |f(h)|\leq C|g(h)|,
\end{equation}
donde $C$ es una constante positiva, para toda $h$ suficientemente
pequeña tal que
\[
\lim_{h\to 0}\frac{|f(h)|}{|g(h)|}<C<\infty
\]
Por ejemplo, uno puede escribir la expansion del seno de la siguiente
manera:
\[
\begin{split}
\sin x=x-\frac{x^3}{3!}+ \mathcal{O}(x^5)\\
\sin x-x+\frac{x^3}{3!}=\mathcal{O}(x^5)
\end{split}
\]

\end{document}